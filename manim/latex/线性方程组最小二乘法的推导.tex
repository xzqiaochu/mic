\documentclass[UTF8]{ctexart}

\usepackage{amsmath}
\usepackage{lmodern}
\usepackage{multirow} % Required for multirows

\title{线性方程组最小二乘法的推导}
\author{Yijun WU}
\date{\today}

\begin{document}
\pagenumbering{gobble}
\maketitle
\newpage
\pagenumbering{arabic}
\section*{推导过程}

对于有$n$个未知数的$m$行线性方程组:

$$
\left\{
    \begin{array}{cc}
        a_{11} x_1 + a_{12} x_2 + \cdots + a_{1n} x_n = b_1 \\
        a_{21} x_1 + a_{22} x_2 + \cdots + a_{2n} x_n = b_2 \\
        \cdots \cdots \\
        a_{m1} x_1 + a_{m2} x_2 + \cdots + a_{mn} x_n = b_m 
    \end{array}
\right.
$$

设该方程组的最优解为:

$$
\left\{
    \begin{array}{cc}
        x_1 = c_1 \\
        x_2 = c_2 \\
        \cdots \cdots\\
        x_m = c_m 
    \end{array}
\right.
$$

则可以定义该线性方程组的方差:

\begin{align*}
    \sigma^2 &= (a_{11} c_1 + a_{12} c_2 + \cdots + a_{1n} c_n - b_1)^2 \\
    & +(a_{21} c_1 + a_{22} c_2 + \cdots + a_{2n} c_n - b_2)^2 \\
    & + \cdots +(a_{m1} c_1 + a_{m2} c_2 + \cdots + a_{mn} c_n - b_m)^2
\end{align*}

\subparagraph*{用矩阵的语言表示:}
记$A\in M_{m\times n}$为$m$行$n$列的矩阵,$\vec{x}$为$n$行的列向量,$\vec{b}$为$m$行的列向量。
则原线性方程组可以表示为:
$$ A\vec{x}=\vec{b} $$

对于任意一个解
$$ x=\left(
    \begin{matrix}
        x_1\\
        x_2\\
        \cdots\\
        x_n
    \end{matrix}
\right) $$

方差可以表示为:
\begin{align*}
    \sigma^2 &= (A\vec{x}-\vec{b})^T (A\vec{x}-\vec{b})\\
    &= (\vec{x}^T A^T-\vec{b}^T) (A\vec{x}-\vec{b})\\
    &= \vec{x}^T A^T A \vec{x} - \vec{x}^T A^T \vec{b} - \vec{b}^T A \vec{x} + \vec{b} \vec{b}^T 
\end{align*}

又由于$ \vec{x}^T A^T \vec{b} $是$1\times 1$矩阵,因此$ \vec{x}^T A^T \vec{b} = (\vec{x}^T A^T \vec{b})^T = \vec{b}^T A \vec{x}$,于是
$$
    \sigma^2 = \vec{x}^T (A^T A) \vec{x} - (2\vec{b}^T A) \vec{x} + \Vert \vec{b} \Vert ^2
$$

不妨令$ S:=A^T A \in M_{n\times n} $为$n$阶方阵,$ \vec{t} := 2 \vec{b}^T A $为$n$列的行向量,$c:= \Vert \vec{b} \Vert ^2$为常数。
则$f(x_1,x_2,\cdots,x_n):=\sigma^2=\vec{x}^T S \vec{t} - \vec{t} \vec{x} + c$,
当$f$取得最值时,有$f'_{x_1}=f'_{x_2}=\cdots=f'_{x_n}=0$。

不妨记$\vec{e_i}$为单位阵$I_n$的第$i$列这个列向量,
记$\vec{S}^{(i)}$为$S$的第$i$列的这个列向量,$\vec{S}_{(i)}$为$S$的第$i$行的这个行向量,
记$t_i$为$\vec{t}$的第$i$个元素。则
\begin{align*}
    \forall i \in \{1,2,3,\cdots,n\},f'_{x_i}&=\frac{\partial\vec{x}^T}{\partial{x_i}}(S\vec{x})+(\vec{x}^T S)\frac{\partial\vec{x}}{\partial{x_i}}-(\vec{t})\frac{\partial\vec{x}}{\partial{x_i}}\\
    &=\vec{e_i}^T (S\vec{x})+(\vec{x}^T S)\vec{e_i}-\vec{t}\vec{e_i}\\
    &=(\vec{e_i}^T S)\vec{x}+\vec{x}^T(S \vec{e_i})-t_i\\
    &=\vec{S}_{(i)}\vec{x}+\vec{x}^T\vec{S}^{(i)}-t_i\\
    &=\vec{S}_{(i)}\vec{x}+(\vec{S}^{(i)})^T\vec{x}-t_i\\
    &=0
\end{align*}
因此有:$ (\vec{S}_{(i)}+(\vec{S}^{(i)})^T)\vec{x}=t_i ,\forall i \in \{1,2,3,\cdots,n\}$
。也即:
\begin{align*}
    \left(\begin{matrix}
        \vec{S}_{(1)}\\
        \vec{S}_{(2)}\\
        \cdots\\
        \vec{S}_{(n)}\\
    \end{matrix}\right)
    \vec{x}+
    \left(\begin{matrix}
        (\vec{S}^{(1)})^T\\
        (\vec{S}^{(2)})^T\\
        \cdots\\
        (\vec{S}^{(n)})^T\\
    \end{matrix}\right)
    \vec{x}=
    \left(\begin{matrix}
        t_1\\
        t_2\\
        \cdots\\
        t_n\\
    \end{matrix}\right)
\end{align*}
由$\vec{S}_{(i)}$和$\vec{S}^{(i)}$和$t_i$的定义知:
\begin{align*}
    && S\vec{x}+S^T\vec{x}=&\vec{t}^T \\
    \Rightarrow&& (S+S^T)\vec{x}=&\vec{t}^T \\
    \Rightarrow&& (AA^T+(AA^T)^T)\vec{x}=&(2\vec{b}^T A)^T \\
    \Rightarrow&& 2 AA^T \vec{x}=&2 A^T \vec{b} \\
    \Rightarrow&& \vec{x} =& (A A^T)^{-1} A^T \vec{b}
\end{align*}

当$\vec{x}=(A A^T)^{-1}A^T\vec{b}$时,线性方程组取最优解。

\paragraph*{其它说明}
有一点需要注意的事情:\\
当$m<=n$时,线性方程的数量不大于未知数的数量,因此存在解使得等号同时成立,也就是$\sigma^2\equiv{0}$。此时该结论没有意义。|\\
当$m>n$,且$A A^T$可逆,也就是$rank(A A^T)=n$时,该结论适用最优解。
当原线性方程组有线性相关的方程时,我也不知道适不适用(笑)

\end{document}